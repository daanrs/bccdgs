\documentclass{article}
\author{Daan Spijkers s1011382 \\ Supervisor: Tom Claassen}
\title{Research Proposal}

\begin{document}

\maketitle

We can view causal inference as taking a system of statistical
independencies, and mining a system of causal relations. These causal
relations are usually represented in a graph.

In an ideal situation, we have a statistical test that determines
whether $x$ and $y$ are independent with 100\% accuracy. Given such a
perfect test, complete algorithms exist; they give the total causal
information possible from that system. Unfortunately, in the real world
100\% accuracy is not possible, and we often have to make do with
insufficient data.

That is why taking realistic data, and optimizing the result is a relevant
problem. It is not always clear how an algorithm performs in these
situations, even if it is complete. One such complete algorithm is BCCD,
which uses a bayesian score to return a more robust and informative result
than comparable procedures.

The topic of my thesis will be an initial attempt to gain some insight
into possible gains, by adding an additional step after BCCD. We will do
this by defining a metric on its derived probabilistic causal statements,
and running a MAG search to optimise it.

The specific research question is: how much can we improve the accuracy of
the resulting PAG from the BCCD algorithm using a greedy MAG search
to optimise its probabilistic causal statements?

For more insight we can vary the causal models we generate, and how much
data we produce. Since BCCD is complete, if we feed it infinite data, we
should see its result converge to the original causal model.

In our research, the main problem that we will need to solve is how
exactly to define a metric on the causal statements. That is where the most
involved effort will have to be.

Although other parts of the problem are simpler, the efficiency of a MAG
search could be a limitation. Generating adjacent graphs is an
expensive procedure, so we might have to restrict ourselves to smaller
graphs.

The original start date was early 2020, with progress being sporadic.
Currently the experiments are finished, and the aim for a complete draft
is May 27th.

\end{document}
