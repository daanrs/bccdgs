% Options for packages loaded elsewhere
\PassOptionsToPackage{unicode}{hyperref}
\PassOptionsToPackage{hyphens}{url}
%
\documentclass[
]{article}
\usepackage{amsmath,amssymb}
\usepackage{lmodern}
\usepackage{iftex}
\ifPDFTeX
  \usepackage[T1]{fontenc}
  \usepackage[utf8]{inputenc}
  \usepackage{textcomp} % provide euro and other symbols
\else % if luatex or xetex
  \usepackage{unicode-math}
  \defaultfontfeatures{Scale=MatchLowercase}
  \defaultfontfeatures[\rmfamily]{Ligatures=TeX,Scale=1}
\fi
% Use upquote if available, for straight quotes in verbatim environments
\IfFileExists{upquote.sty}{\usepackage{upquote}}{}
\IfFileExists{microtype.sty}{% use microtype if available
  \usepackage[]{microtype}
  \UseMicrotypeSet[protrusion]{basicmath} % disable protrusion for tt fonts
}{}
\makeatletter
\@ifundefined{KOMAClassName}{% if non-KOMA class
  \IfFileExists{parskip.sty}{%
    \usepackage{parskip}
  }{% else
    \setlength{\parindent}{0pt}
    \setlength{\parskip}{6pt plus 2pt minus 1pt}}
}{% if KOMA class
  \KOMAoptions{parskip=half}}
\makeatother
\usepackage{xcolor}
\IfFileExists{xurl.sty}{\usepackage{xurl}}{} % add URL line breaks if available
\IfFileExists{bookmark.sty}{\usepackage{bookmark}}{\usepackage{hyperref}}
\hypersetup{
  pdftitle={Thesis},
  pdfauthor={Daan Spijkers},
  hidelinks,
  pdfcreator={LaTeX via pandoc}}
\urlstyle{same} % disable monospaced font for URLs
\usepackage{color}
\usepackage{fancyvrb}
\newcommand{\VerbBar}{|}
\newcommand{\VERB}{\Verb[commandchars=\\\{\}]}
\DefineVerbatimEnvironment{Highlighting}{Verbatim}{commandchars=\\\{\}}
% Add ',fontsize=\small' for more characters per line
\newenvironment{Shaded}{}{}
\newcommand{\AlertTok}[1]{\textcolor[rgb]{1.00,0.00,0.00}{\textbf{#1}}}
\newcommand{\AnnotationTok}[1]{\textcolor[rgb]{0.38,0.63,0.69}{\textbf{\textit{#1}}}}
\newcommand{\AttributeTok}[1]{\textcolor[rgb]{0.49,0.56,0.16}{#1}}
\newcommand{\BaseNTok}[1]{\textcolor[rgb]{0.25,0.63,0.44}{#1}}
\newcommand{\BuiltInTok}[1]{#1}
\newcommand{\CharTok}[1]{\textcolor[rgb]{0.25,0.44,0.63}{#1}}
\newcommand{\CommentTok}[1]{\textcolor[rgb]{0.38,0.63,0.69}{\textit{#1}}}
\newcommand{\CommentVarTok}[1]{\textcolor[rgb]{0.38,0.63,0.69}{\textbf{\textit{#1}}}}
\newcommand{\ConstantTok}[1]{\textcolor[rgb]{0.53,0.00,0.00}{#1}}
\newcommand{\ControlFlowTok}[1]{\textcolor[rgb]{0.00,0.44,0.13}{\textbf{#1}}}
\newcommand{\DataTypeTok}[1]{\textcolor[rgb]{0.56,0.13,0.00}{#1}}
\newcommand{\DecValTok}[1]{\textcolor[rgb]{0.25,0.63,0.44}{#1}}
\newcommand{\DocumentationTok}[1]{\textcolor[rgb]{0.73,0.13,0.13}{\textit{#1}}}
\newcommand{\ErrorTok}[1]{\textcolor[rgb]{1.00,0.00,0.00}{\textbf{#1}}}
\newcommand{\ExtensionTok}[1]{#1}
\newcommand{\FloatTok}[1]{\textcolor[rgb]{0.25,0.63,0.44}{#1}}
\newcommand{\FunctionTok}[1]{\textcolor[rgb]{0.02,0.16,0.49}{#1}}
\newcommand{\ImportTok}[1]{#1}
\newcommand{\InformationTok}[1]{\textcolor[rgb]{0.38,0.63,0.69}{\textbf{\textit{#1}}}}
\newcommand{\KeywordTok}[1]{\textcolor[rgb]{0.00,0.44,0.13}{\textbf{#1}}}
\newcommand{\NormalTok}[1]{#1}
\newcommand{\OperatorTok}[1]{\textcolor[rgb]{0.40,0.40,0.40}{#1}}
\newcommand{\OtherTok}[1]{\textcolor[rgb]{0.00,0.44,0.13}{#1}}
\newcommand{\PreprocessorTok}[1]{\textcolor[rgb]{0.74,0.48,0.00}{#1}}
\newcommand{\RegionMarkerTok}[1]{#1}
\newcommand{\SpecialCharTok}[1]{\textcolor[rgb]{0.25,0.44,0.63}{#1}}
\newcommand{\SpecialStringTok}[1]{\textcolor[rgb]{0.73,0.40,0.53}{#1}}
\newcommand{\StringTok}[1]{\textcolor[rgb]{0.25,0.44,0.63}{#1}}
\newcommand{\VariableTok}[1]{\textcolor[rgb]{0.10,0.09,0.49}{#1}}
\newcommand{\VerbatimStringTok}[1]{\textcolor[rgb]{0.25,0.44,0.63}{#1}}
\newcommand{\WarningTok}[1]{\textcolor[rgb]{0.38,0.63,0.69}{\textbf{\textit{#1}}}}
\usepackage{graphicx}
\makeatletter
\def\maxwidth{\ifdim\Gin@nat@width>\linewidth\linewidth\else\Gin@nat@width\fi}
\def\maxheight{\ifdim\Gin@nat@height>\textheight\textheight\else\Gin@nat@height\fi}
\makeatother
% Scale images if necessary, so that they will not overflow the page
% margins by default, and it is still possible to overwrite the defaults
% using explicit options in \includegraphics[width, height, ...]{}
\setkeys{Gin}{width=\maxwidth,height=\maxheight,keepaspectratio}
% Set default figure placement to htbp
\makeatletter
\def\fps@figure{htbp}
\makeatother
\setlength{\emergencystretch}{3em} % prevent overfull lines
\providecommand{\tightlist}{%
  \setlength{\itemsep}{0pt}\setlength{\parskip}{0pt}}
\setcounter{secnumdepth}{-\maxdimen} % remove section numbering
\ifLuaTeX
  \usepackage{selnolig}  % disable illegal ligatures
\fi

\title{Thesis}
\author{Daan Spijkers}
\date{}

\begin{document}
\maketitle

\hypertarget{thesis-draft}{%
\section{Thesis Draft}\label{thesis-draft}}

\hypertarget{abstract}{%
\subsection{Abstract}\label{abstract}}

\emph{How much can we improve the accuracy of the resulting PAG from the
BCCD algorithm using a greedy MAG search to optimise its probabilistic
causal statements?}

\hypertarget{introduction}{%
\subsection{Introduction}\label{introduction}}

Causal inference is taking a system of statistical independencies, and
mining a system of causal relations. These causal relations we then
represent in a causal graph.

In an ideal situation, we have a statistical test that determines
whether \(x\) and \(y\) are independent with 100\% accuracy. Given such
a perfect test, complete algorithms exist; they give the total causal
information possible from that system. Unfortunately, in the real world
100\% accuracy is not possible, and we often have to make do with
insufficient data.

That is why taking realistic data, and optimizing the result is a
relevant problem. It is not always clear how an algorithm performs in
these situations, even if it is complete. One such complete algorithm is
BCCD{[}@claassenBayesianApproachConstraint2012a{]}, which uses a
bayesian score to return a more robust and informative result than
comparable procedures.

\hypertarget{preliminaries}{%
\subsection{Preliminaries}\label{preliminaries}}

\hypertarget{research}{%
\subsection{Research}\label{research}}

\hypertarget{problem-details}{%
\subsubsection{Problem details}\label{problem-details}}

The topic of my thesis will be an initial attempt to gain some insight
into possible gains, by adding an additional step after BCCD. We will do
this by defining a metric on its derived probabilistic causal
statements, and running a MAG search to optimise it.

The specific research question is: how much can we improve the accuracy
of the resulting PAG from the BCCD algorithm using a greedy MAG search
to optimise its probabilistic causal statements?

For more insight we can vary the causal models we generate, and how much
data we produce. Since BCCD is complete, if we feed it infinite data, we
should see its result converge to the original causal model.

In our research, the main problem that we will need to solve is how
exactly to define a metric on the causal statements. That is where the
most involved effort will have to be.

Although other parts of the problem are simpler, the efficiency of a MAG
search could be a limitation. Generating adjacent graphs is an expensive
procedure, so we might have to restrict ourselves to smaller graphs.

\hypertarget{solution-details}{%
\subsubsection{Solution Details}\label{solution-details}}

\hypertarget{pseudocode}{%
\paragraph{Pseudocode}\label{pseudocode}}

Simple pseudocode for our process is as follows:

\begin{Shaded}
\begin{Highlighting}[]
\KeywordTok{def}\NormalTok{ run(pag):}
\NormalTok{  mag }\OperatorTok{=}\NormalTok{ pag\_to\_mag(pag)}
\NormalTok{  next\_mag }\OperatorTok{=}\NormalTok{ next\_mag(mag)}

  \ControlFlowTok{while}\NormalTok{ score(next\_mag) }\OperatorTok{\textgreater{}}\NormalTok{ score(mag):}
\NormalTok{    mag }\OperatorTok{=}\NormalTok{ next\_mag}
\NormalTok{    next\_mag }\OperatorTok{=}\NormalTok{ next\_mag(mag)}

  \ControlFlowTok{return}\NormalTok{ mag\_to\_pag(mag)}

\KeywordTok{def}\NormalTok{ next\_mag(mag):}
  \ControlFlowTok{return}\NormalTok{ best\_scoring\_mag(adjacent\_mags(mag))}
\end{Highlighting}
\end{Shaded}

Here we see that there are 4 main problems that we need to solve:

\begin{enumerate}
\def\labelenumi{\arabic{enumi}.}
\tightlist
\item
  Transforming a PAG into a MAG.
\item
  Generating adjacent mags.
\item
  Scoring a MAG.
\item
  Transforming a MAG into a PAG
\end{enumerate}

\hypertarget{pag-to-mag}{%
\paragraph{PAG to MAG}\label{pag-to-mag}}

The main difference is circle marks. The way to do this is by first
orienting all semi-arcs into arcs, and then orienting all remaining
edges into a DAG with no unshielded colliders. See Zhang paper.

\hypertarget{mag-to-pag}{%
\paragraph{MAG to PAG}\label{mag-to-pag}}

Turning a MAG back into a PAG is slightly more involved. We use
d-separation and the FCI orientation rules to do so.

\hypertarget{generating-adjacent-mags}{%
\paragraph{Generating adjacent MAGS}\label{generating-adjacent-mags}}

This is the simplest problem to solve. We consider adjacent graphs to be
graphs which have one edge changed compared to the original. Given two
vertices \(u\) and \(v\), there are four possibilities: \begin{align}
  u \rightarrow v \\
  u \leftarrow v \\
  u \leftrightarrow v\\
  (u, v) \notin E
\end{align} Remembering that we ignored selection bias, and so there are
no undirected edges. Our original graph has one of these four. All
adjacent MAGS can then be easily generated by adding a graph which has
one of the other 3 possibilities.

The real issue comes up when we want to check whether this MAG is also
valid; that is whether it has any almost directed cycles.

\hypertarget{scoring-a-mag}{%
\paragraph{Scoring a MAG}\label{scoring-a-mag}}

Scoring is not necessarily the most difficult part, but it is more
unique to our problem, and did not have a readily available
implementation. We have implemented 3 checks:

\begin{enumerate}
\def\labelenumi{\arabic{enumi}.}
\tightlist
\item
  Ancestor(x, y)
\item
  Edge(x, y)
\item
  Cofounder(x, y)
\end{enumerate}

\hypertarget{benchmarks}{%
\paragraph{Benchmarks}\label{benchmarks}}

While the bccd portion, as well as the fci portion are both measured in
seconds, my part is only measured in miliseconds.

\hypertarget{results}{%
\subsubsection{Results}\label{results}}

\includegraphics{./lib/nodes_pag.pdf}
\includegraphics{./lib/nodes_causal.pdf}
\includegraphics{./lib/skel_pag.pdf}
\includegraphics{./lib/skel_causal.pdf}
\includegraphics{./lib/sparsity_pag.pdf}
\includegraphics{./lib/sparsity_causal.pdf}

\hypertarget{related-work}{%
\subsection{Related Work}\label{related-work}}

\hypertarget{conclusion}{%
\subsection{Conclusion}\label{conclusion}}

\hypertarget{references}{%
\subsection{References}\label{references}}

\end{document}
